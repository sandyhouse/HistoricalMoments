\documentclass[lang=cn,newtx,10pt,scheme=chinese]{elegantbook}

\title{时光剪影:我的全球观察与见证}
%\subtitle{当代历史的个人记录与评论}

\author{Long Li}
%\institute{佚名}
\date{2024/08/19}
\version{0.1}
\bioinfo{书籍简介}{本书是从个人视角记录时代脉动的独特作品。}

%\extrainfo{注意:本模板自 2023 年 1 月 1 日开始,不再更新和维护!}

\setcounter{tocdepth}{3}

%\logo{logo-blue.png}
\cover{images/cover.jpeg}

% 本文档命令
\usepackage{array}
\newcommand{\ccr}[1]{\makecell{{\color{#1}\rule{1cm}{1cm}}}}

% 修改标题页的橙色带
\definecolor{customcolor}{RGB}{32,178,170}
\colorlet{coverlinecolor}{customcolor}
\usepackage{cprotect}

\addbibresource[location=local]{reference.bib} % 参考文献,不要删除

\begin{document}

\maketitle
\frontmatter

\tableofcontents

\mainmatter

\chapter{序言}
%《时光剪影:我的全球观察与见证》是一部从个人视角记录时代脉动的独特作品。作者通过自己的眼睛和心灵,捕捉了国内外的重大事件,并将这些瞬间化为深刻的思考与见证。书中穿插着个人体验与全球动态的交织,既有对历史的铭刻,也有对未来的展望。在纷繁复杂的世界中,作者以独到的见解为读者呈现了一幅跨越时空的全景图,带领读者穿越时光,回顾那些塑造我们当下与未来的重要时刻。

生活在这个信息爆炸的时代,世界的脉动变得前所未有的清晰,每一天都仿佛是历史书写的新篇章。面对不断涌现的新闻和事件,我们每个人都是见证者,同时也是历史的参与者。正是怀着这样的感受,我开始了《时光剪影:我的全球观察与见证》的写作。

**记录时代,思考未来**  
本书的核心目的在于记录和反思。时代的洪流裹挟着无数重要的瞬间,有些我们目睹了却很快被遗忘,有些则深深嵌入我们的记忆中,久久不能散去。我希望通过个人的视角,把这些闪光的片段串联成线,构建出一幅时代的剪影。在这些事件中,不仅有全球视野下的动荡与变迁,也有发生在我们身边的每一个微小但深刻的变化。

**个人与历史的对话**  
这不是一部简单的新闻记录,也不仅仅是事实的堆积。每一个事件在我的书写中,都是一场个人与历史的对话。我试图透过现象看本质,剖析事件背后的驱动力和影响力,同时将我的思考、疑问与感悟融入其中。这样的叙述,既是对事件的忠实记录,也是对自己内心世界的揭示。

**全球视野与本土情怀**  
在这本书中,国际与国内的事件交相辉映。全球视野让我们看见了大国博弈、文化碰撞与技术革命,而本土情怀则使我们感受到社会变迁、民生改善与文化传承。这些事件在时间与空间的交汇中,共同塑造了我们的现实生活,也影响着未来的走向。

**展望与启迪**  
写作的过程也是思考与自我探索的过程。回顾这些年来的风云变幻,我更加确信,每一个时代都有其独特的使命与挑战,而我们所经历的正是这样一个充满机遇与风险的时代。希望本书能带给读者一些启迪,促使大家在回顾过往的同时,更多地思考未来。毕竟,历史的记录不仅是为了铭记,更是为了展望。


\chapter{新冠疫情后的~2024}

从中国的角度来看,2024年是新冠疫情封控全面放开后的第一年。2024年本是充满希望的一年,因为摆脱了疫情的束缚,我们终于又可以展开手脚,准备大干一场。。。

\end{document}