\chapter{2025}

\section{习近平 2025 年新年贺词}
新华社北京12月31日电 新年前夕,国家主席习近平通过中央广播电视总台和互联网,发表了二〇二五年新年贺词。全文如下:

大家好!时间过得很快,新的一年即将到来,我在北京向大家致以美好的祝福!

2024年,我们一起走过春夏秋冬,一道经历风雨彩虹,一个个瞬间定格在这不平凡的一年,令人感慨、难以忘怀。

我们积极应对国内外环境变化带来的影响,出台一系列政策“组合拳”,扎实推动高质量发展,我国经济回暖向好,国内生产总值预计超过130万亿元。粮食产量突破1.4万亿斤,中国碗装了更多中国粮。区域发展协同联动、积厚成势,新型城镇化和乡村振兴相互融合、同频共振。绿色低碳发展纵深推进,美丽中国画卷徐徐铺展。

我们因地制宜培育新质生产力,新产业新业态新模式竞相涌现,新能源汽车年产量首次突破1000万辆,集成电路、人工智能、量子通信等领域取得新成果。嫦娥六号首次月背采样,梦想号探秘大洋,深中通道踏浪海天,南极秦岭站崛起冰原,展现了中国人逐梦星辰大海的豪情壮志。

今年,我到地方考察,看到大家生活多姿多彩。天水花牛苹果又大又红,东山澳角村渔获满舱。麦积山石窟“东方微笑”跨越千年,六尺巷礼让家风代代相传。天津古文化街人潮熙攘,银川多民族社区居民亲如一家。对大家关心的就业增收、“一老一小”、教育医疗等问题,我一直挂念。一年来,基础养老金提高了,房贷利率下调了,直接结算范围扩大方便了异地就医,消费品以旧换新提高了生活品质……大家的获得感又充实了许多。

巴黎奥运赛场上,我国体育健儿奋勇争先,取得境外参赛最好成绩,彰显了青年一代的昂扬向上、自信阳光。海军、空军喜庆75岁生日,人民子弟兵展现新风貌。面对洪涝、台风等自然灾害,广大党员干部冲锋在前,大家众志成城、守望相助。无数劳动者、建设者、创业者,都在为梦想拼搏。我为国家勋章和国家荣誉称号获得者颁奖,光荣属于他们,也属于每一个挺膺担当的奋斗者。

当今世界变乱交织,中国作为负责任大国,积极推动全球治理变革,深化全球南方团结合作。我们推进高质量共建“一带一路”走深走实,成功举办中非合作论坛北京峰会,在上合、金砖、亚太经合组织、二十国集团等双边多边场合,鲜明提出中国主张,为维护世界和平稳定注入更多正能量。

我们隆重庆祝新中国成立75周年,深情回望共和国的沧桑巨变。从五千多年中华文明的传承中一路走来,“中国”二字镌刻在“何尊”底部,更铭刻在每个华夏儿女心中。党的二十届三中全会胜利召开,吹响进一步全面深化改革的号角。我们乘着改革开放的时代大潮阔步前行,中国式现代化必将在改革开放中开辟更加广阔的前景。

2025年,我们将全面完成“十四五”规划。要实施更加积极有为的政策,聚精会神抓好高质量发展,推动高水平科技自立自强,保持经济社会发展良好势头。当前经济运行面临一些新情况,有外部环境不确定性的挑战,有新旧动能转换的压力,但这些经过努力是可以克服的。我们从来都是在风雨洗礼中成长、在历经考验中壮大,大家要充满信心。

家事国事天下事,让人民过上幸福生活是头等大事。家家户户都盼着孩子能有好的教育,老人能有好的养老服务,年轻人能有更多发展机会。这些朴实的愿望,就是对美好生活的向往。我们要一起努力,不断提升社会建设和治理水平,持续营造和谐包容的氛围,把老百姓身边的大事小情解决好,让大家笑容更多、心里更暖。

在澳门回归祖国25周年之际,我再到濠江之畔,新发展新变化令人欣喜。我们将坚定不移贯彻“一国两制”方针,保持香港、澳门长期繁荣稳定。两岸同胞一家亲,谁也无法割断我们的血脉亲情,谁也不能阻挡祖国统一的历史大势!

世界百年变局加速演进,需要以宽广胸襟超越隔阂冲突,以博大情怀关照人类命运。中国愿同各国一道,做友好合作的践行者、文明互鉴的推动者、构建人类命运共同体的参与者,共同开创世界的美好未来。

梦虽遥,追则能达;愿虽艰,持则可圆。中国式现代化的新征程上,每一个人都是主角,每一份付出都弥足珍贵,每一束光芒都熠熠生辉。

河山添锦绣,星光映万家。让我们满怀希望,迎接新的一年。祝祖国时和岁丰、繁荣昌盛!祝大家所愿皆所成,多喜乐、长安宁!

\section{跨年与维稳}
维稳