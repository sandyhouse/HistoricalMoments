\chapter{2025}

\section{习近平 2025 年新年贺词}
新华社北京12月31日电 新年前夕,国家主席习近平通过中央广播电视总台和互联网,发表了二〇二五年新年贺词。全文如下:

大家好!时间过得很快,新的一年即将到来,我在北京向大家致以美好的祝福!

2024年,我们一起走过春夏秋冬,一道经历风雨彩虹,一个个瞬间定格在这不平凡的一年,令人感慨、难以忘怀。

我们积极应对国内外环境变化带来的影响,出台一系列政策“组合拳”,扎实推动高质量发展,我国经济回暖向好,国内生产总值预计超过130万亿元。粮食产量突破1.4万亿斤,中国碗装了更多中国粮。区域发展协同联动、积厚成势,新型城镇化和乡村振兴相互融合、同频共振。绿色低碳发展纵深推进,美丽中国画卷徐徐铺展。

我们因地制宜培育新质生产力,新产业新业态新模式竞相涌现,新能源汽车年产量首次突破1000万辆,集成电路、人工智能、量子通信等领域取得新成果。嫦娥六号首次月背采样,梦想号探秘大洋,深中通道踏浪海天,南极秦岭站崛起冰原,展现了中国人逐梦星辰大海的豪情壮志。

今年,我到地方考察,看到大家生活多姿多彩。天水花牛苹果又大又红,东山澳角村渔获满舱。麦积山石窟“东方微笑”跨越千年,六尺巷礼让家风代代相传。天津古文化街人潮熙攘,银川多民族社区居民亲如一家。对大家关心的就业增收、“一老一小”、教育医疗等问题,我一直挂念。一年来,基础养老金提高了,房贷利率下调了,直接结算范围扩大方便了异地就医,消费品以旧换新提高了生活品质……大家的获得感又充实了许多。

巴黎奥运赛场上,我国体育健儿奋勇争先,取得境外参赛最好成绩,彰显了青年一代的昂扬向上、自信阳光。海军、空军喜庆75岁生日,人民子弟兵展现新风貌。面对洪涝、台风等自然灾害,广大党员干部冲锋在前,大家众志成城、守望相助。无数劳动者、建设者、创业者,都在为梦想拼搏。我为国家勋章和国家荣誉称号获得者颁奖,光荣属于他们,也属于每一个挺膺担当的奋斗者。

当今世界变乱交织,中国作为负责任大国,积极推动全球治理变革,深化全球南方团结合作。我们推进高质量共建“一带一路”走深走实,成功举办中非合作论坛北京峰会,在上合、金砖、亚太经合组织、二十国集团等双边多边场合,鲜明提出中国主张,为维护世界和平稳定注入更多正能量。

我们隆重庆祝新中国成立75周年,深情回望共和国的沧桑巨变。从五千多年中华文明的传承中一路走来,“中国”二字镌刻在“何尊”底部,更铭刻在每个华夏儿女心中。党的二十届三中全会胜利召开,吹响进一步全面深化改革的号角。我们乘着改革开放的时代大潮阔步前行,中国式现代化必将在改革开放中开辟更加广阔的前景。

2025年,我们将全面完成“十四五”规划。要实施更加积极有为的政策,聚精会神抓好高质量发展,推动高水平科技自立自强,保持经济社会发展良好势头。当前经济运行面临一些新情况,有外部环境不确定性的挑战,有新旧动能转换的压力,但这些经过努力是可以克服的。我们从来都是在风雨洗礼中成长、在历经考验中壮大,大家要充满信心。

家事国事天下事,让人民过上幸福生活是头等大事。家家户户都盼着孩子能有好的教育,老人能有好的养老服务,年轻人能有更多发展机会。这些朴实的愿望,就是对美好生活的向往。我们要一起努力,不断提升社会建设和治理水平,持续营造和谐包容的氛围,把老百姓身边的大事小情解决好,让大家笑容更多、心里更暖。

在澳门回归祖国25周年之际,我再到濠江之畔,新发展新变化令人欣喜。我们将坚定不移贯彻“一国两制”方针,保持香港、澳门长期繁荣稳定。两岸同胞一家亲,谁也无法割断我们的血脉亲情,谁也不能阻挡祖国统一的历史大势!

世界百年变局加速演进,需要以宽广胸襟超越隔阂冲突,以博大情怀关照人类命运。中国愿同各国一道,做友好合作的践行者、文明互鉴的推动者、构建人类命运共同体的参与者,共同开创世界的美好未来。

梦虽遥,追则能达;愿虽艰,持则可圆。中国式现代化的新征程上,每一个人都是主角,每一份付出都弥足珍贵,每一束光芒都熠熠生辉。

河山添锦绣,星光映万家。让我们满怀希望,迎接新的一年。祝祖国时和岁丰、繁荣昌盛!祝大家所愿皆所成,多喜乐、长安宁!

\section{跨年与维稳}

2025 年元旦前夕,中国多地官方宣布不组织跨年夜活动,其中包括人气最高的上海外滩以及广州、武汉等城市热门跨年景点。中国应急管理部要求各地加强跨年夜活动安全管控,防范发生事故。

综合九派新闻等陆媒报导,上海警方26日宣布,上海外滩、北外滩和小陆家嘴地区31日晚上不展演主题光影秀,不组织零点倒数计时活动。

在此之前,广州多区公安发布通告,广州塔、太古仓、花城广场、海心沙亚运公园、北京路商圈、海珠广场、越秀公园等公共场所,跨年夜没有组织任何形式的大型群众性活动。

南京鸡鸣寺官方微博28日发布通知称,元旦期间不举行撞钟跨年等活动。

武汉警方也发布提醒,江汉关、汉口江滩及江汉路步行街等公共场所,31日没有组织任何形式的大型群众活动。同时提醒市民,请尽量不要前往这些人群高度聚集的公共场所。江汉关钟楼明晚6时后停止敲钟,「两江四岸」灯光秀也无零点倒数计时活动。

多地警方下令,跨年夜热门地点禁止燃放鞭炮、禁飞无人机以及贩售施放天灯、气球。

中国国家应急管理部28日发布「元旦假期安全提示」,要求各地关注集会活动安全,表示元旦假期,节庆、旅游、娱乐等活动增多,人流物流聚集,需加强「跨年夜」等重大集会活动安全管控,防范人员密集场所火灾和踩踏等事故。

多地应急管理部门近日相继展开「跨年夜活动」安全生产督察。

不过在微博相关新闻讨论区,大陆网友多表示,官方不办跨年夜活动,还是照样去跨年,民众会自己喊倒数计时。

社群讯息显示,尽管中国多地官方不办跨年夜活动,但民间经营的游乐场、商城、景区,仍有相当丰富的烟火秀、演唱会等活动。 

大陆的网友“北风过熙”对此讥讽表示,“这真的是管控上瘾了。”

旅澳的历史学者李元华表示,“中共自己实际上是心虚,它知道经济不景气,老百姓怨声很多。它害怕这种群众性的聚集活动,因为它觉得是不可操控的。”

他进一步表示,如果在这个活动中大家打出了自己的心声,恐因此而引起社会大的动荡,它就觉得没有保证。中共因为没有自信嘛,习的下属就是为了保证一个表面的平安,它宁可用这种强暴式的做法,一级一级地压下来。“它不管过去那种假繁荣、盛世这种场面都不要了,连这一层伪装都不要了,它非常恐惧老百姓把这种庆祝活动转向反共的,实际上说反抗中共暴政的事件。”

北京有市民李先生表示,北京街头新年烟花也不给放,我们只能偷偷地放。当局现在异常恐惧民间维权抗争,所以新年北京街头这段时间特别多的是群防群治这种标语。

此前总部位于香港的中国劳工通讯(China Labour Bulletin, CLB)公布了最新数据显示,2023年中国工人维权运动风起云涌,截至2023年12月27日,中国的工人维权事件已达1,919宗,已超过以往三年的数字总和。而这只是冰山一角。

前北京律师、民阵加拿大主席赖建平曾表示:“现在中国遍地干柴烈火,其中失业大潮、下岗大潮遍及各个行业各个领域,所以维权事件此起彼伏,到处都是狼烟、烽火,在这种情况之下,中共完全没有可能都去镇压。”他还强调,“面对遍地开花的维权活动,中共是没有办法处置的,所以某种程度上,对它的政权是构成了极大的一个威胁,中共弄不好就会葬送在中国人民的维权运动之中。”

\section{2025年股汇}

进入2025年,A股并没有迎来开门红,在两个交易日里沪指跌破3300点,接近10月以来的震荡区间下沿,A股有近5000家个股下跌,各主要宽基指数、风格指数和行业指数普跌。具体来说,2024年12月31日,节前主要跌的是微盘股,2025年1月2日过节回来的第一天,开门眼前一片黑,2024年1月3日沪指跌超1.5个点,深成指跌了快2个点,创业板跌超过2个点。近三个交易日,沪指累计下跌快6个点,跌幅接近200点。

汇率方面,2025年1月3日,在岸人民币兑美元汇率跌破7.30关口,为2023年11月以来首次,引发市场关注。

\section{台北跨年播放央视统战片与国民党的“魅共”}

台北市政府跨年晚会吸引了超过22万人参加。然而,就在跨年晚会结束后不久,市府外大屏幕上竟在播放中国央视的跨年节目!画面被上传到脸书社团后引发热议,文总秘书长李厚庆更批评台北市府,“为什么要放共产党的节目?”

不过有网友却认为,这就像有的新闻也会转播雪梨歌剧院跨年烟火一样;但这并非新闻频道,“是中华民国首都台北市跨年晚会的大萤幕。”

\section{2025中国股市}

中国股市2025年开局不利,沪深300指数今天(2日)下挫2.9\%至3820点,这是2016年以来最糟的年度首个交易日表现,凸显中国去年秋季以来的涨势并不稳固,任何风吹草动的消息面造成的影响,都难被中南海的言辞挽救。

彭博报道,这是中国主要指数连续第2个交易日走跌,就在2024年最后1个交易日、12月31日时,沪深300指数跌破60日均线,导致股指卖压沉重,而在今天,财新/标普(Caixin/S\&P)发布低于预期的制造业采购经理人指数(PMI),导致对经济振兴缺乏信心,以及发动美中贸易战的川普即将在本月20日就职美国总统等负面情绪全部接踵而来,导致股市大跌。

\section{俄乌战争}

\subsection{乌克兰中断俄罗斯天然气过境输送管道}

2019年12月,俄罗斯天然气工业股份公司与乌克兰石油天然气公司签署了关于经乌克兰领土输送天然气的协议,协议有效期5年,于2024年12月31日到期。此前,乌克兰频频表态“停气”“不续约”,引发斯洛伐克、匈牙利等欧盟成员国对于能源供应的担忧。

俄罗斯天然气工业股份公司1月1日发表声明说,因过境协议到期,自莫斯科时间1月1日早8时起终止过境乌克兰向欧洲输送天然气。同日,乌克兰能源部表示,基辅时间1月1日早7时,乌克兰停止了俄罗斯天然气过境输送服务。

对于此前通过俄乌这条管线获取俄天然气的欧洲国家会产生怎样的影响?

俄罗斯副总理亚历山大·诺瓦克日前表示,2024年1至11月,俄罗斯经管道输送至欧洲的天然气,以及出口欧洲的液化天然气总量超过500亿立方米,同比增长18\%至20\%。据路透社测算,这其中约有150亿立方米过境乌克兰,输送至斯洛伐克等欧洲国家,占欧洲天然气总体供应量的5\%左右。因此,俄天然气停止过境乌克兰会对斯洛伐克以及部分中东欧国家产生直接影响。而对匈牙利而言,2021年,该国与俄罗斯天然气工业股份公司签署了一份为期15年的购气协议,俄方经绕过乌克兰的“土耳其溪”天然气管道,向匈牙利输送该国使用的大部分天然气。因此,俄停止通过乌克兰向欧洲供应天然气对匈牙利影响有限。

除了斯洛伐克以外,其他国家预见到接下来俄乌之间的天然气合作很难再持续,所以寻找了其他一些替代方案。奥地利政府此前一再表示,将通过从德国、意大利进口天然气来进行替代,不会出现天然气短缺,但奥地利国内舆论认为,天然气价格上涨恐无法避免。奥地利燃气管理公司1月1日表示,已记录到相关管网的天然气交付量降至为零,但奥地利能源行业已做好准备,奥地利国内天然气供应将继续得到保障。不过奥地利舆论认为,虽然奥地利天然气供应得到保障,但价格或将出现上涨。奥地利电力和天然气行业监管机构的专家预计,短期内每兆瓦时天然气价格将上涨3至10欧元。奥地利《维也纳日报》指出,鉴于目前欧盟天然气价格本来就比俄乌冲突爆发前高出很多,即使不再出现2022年那样的价格飙升,哪怕10\%的上涨也是一笔不菲的开支,这可能会给家庭和企业带来进一步的压力,并再次推高通胀。

根据美国经济指标网站显示,受供应担忧和寒冷天气导致需求上升的推动,欧洲天然气期货价格已在当地时间1月1日攀升至每兆瓦时50.53欧元,为近一个月以来的最高水平。在过去一年中,欧洲天然气期货价格大幅上涨,且呈现出持续走高的态势,尤其是随着俄罗斯停止通过乌克兰供气,可能给本就紧张的市场带来更多压力。

崔洪建指出,乌克兰方面切断和俄罗斯的天然气联系,不仅影响到了欧洲的能源安全,更重要的是让欧洲和俄罗斯之间关系,似乎更难再回到从前。因为这样下去的话,现在欧洲和俄罗斯之间,不仅在战略安全上处于对抗的态势;在经济上,存在逐渐切割的状态;同时在能源上,现在几乎已经没有俄罗斯和欧洲之间直接的天然气联系。

目前乌克兰方面不顾斯洛伐克等国反对,结束和俄罗斯的协议,并停止了俄罗斯天然气过境输送服务。此举是出于什么样的考虑?

第一,从乌克兰方面官方表态来看,是出于国家安全的考虑。尽管切掉切断这条管线之后,乌克兰方面也会蒙受一定的经济损失,因为此前收取了过境费,同时能够相对优惠地获得俄罗斯的天然气供应,这样一些好处都已经不复存在,但是显然出于政治和安全上的考虑,乌克兰方面仍然要采取这样的行动。第二,乌克兰希望此举迫使一些国家切断与俄联系。对乌克兰方面来说,希望通过这个举措进一步对欧洲一些国家采取相应措施,这样迫使这些国家进一步切断和俄罗斯的联系。尤其是我们看到这次“断气”事件以后,主要影响和打击的是斯洛伐克。自从菲佐总理上台以后,无论是对俄罗斯的政策、对乌克兰危机的立场,相比此前的斯洛伐克政府有很大的变化和调整。所以在这样的情况下,乌克兰方面这么做,显然是想让斯洛伐克方面在逐渐切断和俄罗斯的能源联系之后,在对乌克兰的立场上出现进一步的变化。第三,乌希望获美支持,称准备采购美国页岩气。最近两天的消息,进一步证实了乌克兰这么做,显然还有更深远的一些考虑。就在这两天,乌克兰方面宣布,准备开始购买来自美国的页岩气。对当前和今后一段时期的乌克兰来说,为了确保美国尤其是在特朗普上台之后,能够持续全面给乌克兰提供各种支持,尤其是军事援助,显然要让美国尝到一些甜头。通过切断和俄罗斯的能源联系,然后转而通过购买美国页岩气向特朗普示好,这样来换取特朗普接下来可能在所谓的政治解决方向上做出一些有利于乌克兰的举动。所以无论是从针对俄罗斯、针对斯洛伐克的角度,还是针对未来的美国政府,我想这一次乌克兰采取的“断气”行为,实际上是想收到“一石三鸟”的效果。

俄乌冲突爆发后,欧盟追随美国对俄罗斯能源产品施加多轮制裁,并制定了在2027年前与俄“能源脱钩”的目标。舆论指出,两年多来欧盟逐步减少对俄能源依赖,自身却遭受了这一系列政策的反噬,而美国则从中坐收渔利、成为最大赢家。

媒体和分析人士认为,如果俄乌无法在短期内解决天然气过境问题,部分欧盟成员国将进一步受到能源短缺压力,俄罗斯每年将失去数十亿美元的收入,乌克兰每年也将损失约8亿美元的过境费,只有美国从中坐收渔利。俄乌冲突爆发后,美国一跃成为世界第一大液化天然气出口国和欧盟最大的液化天然气供应国。据美国商业内幕网站2022年报道,美国能源公司只需要花6000万美元就可以将一艘液化天然气运输船装满,而欧洲的收购价却高达2.75亿美元。除去运输等成本,美国的每船液化天然气都可以从欧洲赚取超过1.5亿美元。

美国当选总统特朗普日前表示,欧盟必须通过大规模购买美国石油和天然气来弥补对美巨额贸易顺差,否则将面临全面关税。分析人士认为,此举试图将欧盟在能源供应领域与美国深度绑定,以更高价格向欧盟出售能源,一方面加强对欧洲能源市场的控制,另一方面也借机进一步巩固美国在欧洲地缘政治中的影响力。

\section{韩国总统尹锡悦逮捕事件}

2024年12月3日,尹锡悦宣布紧急戒严令,指反对派控制国会、同情朝鲜,并透过反国家活动瘫痪政府。随后国会投票通过解除戒严令。2024年12月7日,因戒严风波导致的弹劾案因仅得到195票,未达法定标准200票弹劾失败。在野党宣布将发起第二次弹劾。此后尹锡悦被控犯有内乱罪,并被下令禁止出境,成为韩国历史上首位被禁止出境的在任总统。2024年12月10日,韩国国会通过议案要求立即逮捕尹锡悦等8名与戒严令有关的人物。根据12月13日发布的一项民调显示,尹锡悦的支持率降至11\%,不支持率升至85\%,刷新其就任总统以来的新低和新高。韩国国会14日下午就针对总统尹锡悦的第二次弹劾动议案进行表决,以204票赞成通过了该项动议案。表决结果显示,当天共有300名议员参加表决,其中204人赞成、85人反对,弃权3票,无效8票,赞成票超过了通过弹劾动议案所需票数(即国会议员总人数的三分之二),尹锡悦当天起被停职。

2024年12月30日,韩国联合调查本部申请拘捕总统尹锡悦。尹锡悦在过去两周曾经三次拒绝接受调查。12月31日,韩国法院同意以涉嫌内乱为由对尹锡悦发布逮捕令。

2025年1月3日,韩国高级公职者犯罪调查处来到韩国总统室逮捕尹锡悦,但遭到总统警卫处阻拦,双方发生对峙,最后以公调处暂时撤退结束。

2025年1月6日,即针对尹锡悦的逮捕令有效期的最后一日,韩国高级公职人员犯罪调查处(公调处)称,将向法院申请延长对被停职总统尹锡悦的逮捕令有效期。